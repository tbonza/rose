\documentclass{beamer}
 
\usepackage[utf8]{inputenc}
 
 
%Information to be included in the title page:
\title{Semi-supervised Learning with Deep Generative Models}
\subtitle{Kingma et. al. (2014)}
\author{Tyler Brown}
\institute{CS 7180}
\date{}
 
\begin{document}
 
\frame{\titlepage}
 
\begin{frame}
  \frametitle{Motivating Question}
  How can we model data of increasing size when obtaining label 
  information is difficult?
\end{frame}

\begin{frame}
  \frametitle{High-level Answer}

  We can estimate missing label information by
  using a probabilistic model.
  
\end{frame}

\begin{frame}
  \frametitle{Specifying the Probabilistic Model for Missing Labels}

    \begin{itemize}
  \item Data appears as pairs $(\mathbf{X}, \mathbf{Y}) =
    \{(\mathbf{x}_1, y_1), ..., (\mathbf{x}_N, y_N)\}$
  with the $i$-th observation $x_i \in \mathbb{R}^D$ and a
  corresponding class label $y_i \in \{1, ..., L\}$

  \begin{itemize}
\item Each pair of observations $(x_i,y_i)$ has a corresponding
  latent variable $z_i$

\item Empirical distribution over the labelled and unabelled subsets
  is referred to as $\tilde{p_l}(\mathbf{x}, y)$ and
  $\tilde{p_u}(\mathbf{x})$
  \end{itemize}
\item We can estimate $y_i$ for $x_i$ in distribution
  $\tilde{p_u}(\mathbf{x})$ by finding the maximum
  probability of $p(y_i)$ by using a set of features
  related to $z_i$ and a predictive model
  \begin{enumerate}
  \item \textbf{Latent-feature discriminative model (M1)}
  \item \textbf{Generative semi-supervised model (M2)}
  \item \textbf{Stacked generative semi-supervised model (M1+M2)}
  \end{enumerate}
  \end{itemize}
  \end{frame}

\begin{frame}
  \frametitle{Bayes Rule is used when specifying M1 \& M2}

  \begin{align*}
    p(x,y) &= p(x)p(y|x) \\
    &= p(y)p(x|y) \\
    p(x|y) &= \frac{p(x)p(y|x)}{p(y)}
  \end{align*}

  We need to find an inferred posterior distribution $p_\theta (.)$ \newline
  for M1 \footnotemark
  \[p_\theta (\mathbf{x}|\mathbf{z})\] and
  M2 \footnotemark \[p_\theta (\mathbf{x}|y, \mathbf{z})\]


  \footnotetext[1]{Kingma et. al. (2014) equation (1)}
  \footnotetext[2]{Kingma et. al. (2014) equation (2)}
  
  \end{frame}
 
\end{document}
