%%%%%%%%%%%%%%%%%%%%%%%%%%%%%%%%%%%%%%%%%%%%%%%%
% 1. Document class 
\documentclass[a4paper,12pt]{article} % This defines the style of your paper
%%%%%%%%%%%%%%%%%%%%%%%%%%%%%%%%%%%%%%%%%%%%%%%%
% 2. Packages
\usepackage[top = 2.5cm, bottom = 2.5cm, left = 2.5cm, right = 2.5cm]{geometry} 
\usepackage[T1]{fontenc}
\usepackage[utf8]{inputenc}
\usepackage{multirow} % Multirow is for tables with multiple rows within one cell.
\usepackage{booktabs} % For even nicer tables.
\usepackage{graphicx} 
\usepackage{setspace}
\setlength{\parindent}{0in}
\usepackage{float}
\usepackage{fancyhdr}
\usepackage{titlesec}
\usepackage{url}
\usepackage{amsmath,amssymb,amsthm,bm}
\usepackage{subcaption}

\titleformat*{\section}{\large\bfseries}
\titleformat*{\subsection}{\bfseries}
%%%%%%%%%%%%%%%%%%%%%%%%%%%%%%%%%%%%%%%%%%%%%%%%
% 3. Header (and Footer)
\pagestyle{fancy} % With this command we can customize the header style.
\fancyhf{} % This makes sure we do not have other information in our header or footer.
\lhead{\footnotesize  CS 7180}% \lhead puts text in the top left corner. \footnotesize sets our font to a smaller size.
%\rhead works just like \lhead (you can also use \chead)
\rhead{\footnotesize Assignment 2} %<---- Fill in your lastnames.
% Similar commands work for the footer (\lfoot, \cfoot and \rfoot).
% We want to put our page number in the center.
\cfoot{\footnotesize \thepage} 

\begin{document}
\thispagestyle{empty} % This command disables the header on the first page. 

\begin{tabular}{p{15.5cm}} % This is a simple tabular environment to align your text nicely 
{\large \bf CS 7180 Special Topics in AI: Deep Learning} \\
Northeastern University, Spring 2019 \\
\hline % \hline produces horizontal lines.
\end{tabular} % Our tabular environment ends here.

\vspace*{0.3cm} % Now we want to add some vertical space in between the line and our title.

\begin{center} % Everything within the center environment is centered.
    {\Large \bf Assignment 1} % <---- Don't forget to put in the right number
    \vspace{2mm}
    
        % YOUR NAMES GO HERE
    {Name: Tyler Brown UID: 001684955}
\end{center} 
%
\vspace{0.2cm}

\section{Problem 1}

Assignment 1 requested that we use PyTorch to build a RNN to generate
poetry based on 154 sonnets from William Shakespeare. A sonnet is thought
to be amenable to generative modeling because they follow a specific format.
The approach I chose is based on a PyTorch tutorial \cite{practica59:online}
on generating Shakespeare using a character-level RNN. Improvements to this
model were based on Deep-speare by Lau et. al. (2018) \cite{lau2018deepspeare}.
I review the pre-processing, RNN, and RNN improvements I took during this
assignment. One of my key take-aways is that feature engineering is still an
important part of model performance for this application of deep learning.

\subsection{Pre-processing}

\subsection{RNN}

\subsection{Improving RNNs}

\subsection{Discussion}

I would have gone deeper.


\bibliographystyle{unsrt}
\bibliography{references}


\end{document}
